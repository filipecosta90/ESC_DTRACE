
% !TEX encoding = UTF-8 Unicode



\documentclass[a4paper]{article}
\usepackage{dirtytalk}
\usepackage{geometry}
\geometry{
 a4paper,
 total={150mm,247mm},
 left=30mm,
 top=30mm,
 }
\usepackage{quotchap}

\usepackage{graphicx}
\usepackage{epstopdf}
\DeclareGraphicsExtensions{.eps}
\usepackage{url}
\usepackage{multicol}% http://ctan.org/pkg/multicols
\usepackage{epigraph}
\setlength{\epigraphwidth}{.85\textwidth}

\usepackage[utf8x]{inputenc} 

\usepackage{array}
\newcolumntype{L}[1]{>{\raggedright\let\newline\\\arraybackslash\hspace{0pt}}m{#1}}
\newcolumntype{C}[1]{>{\centering\let\newline\\\arraybackslash\hspace{0pt}}m{#1}}
\newcolumntype{R}[1]{>{\raggedleft\let\newline\\\arraybackslash\hspace{0pt}}m{#1}}
\usepackage{subcaption}
\usepackage[utf8]{inputenc}
\usepackage[portuguese]{babel}
\usepackage{listings,mdframed}

\usepackage{textcomp}
\usepackage{hyperref}

\usepackage{float}
\usepackage{listings}

\lstdefinestyle{esc} {
language=c,
	keywordstyle=\bfseries\ttfamily\color[rgb]{0,0,1},
	identifierstyle=\ttfamily,
	commentstyle=\color[rgb]{0.133,0.545,0.133},
	stringstyle=\ttfamily\color[rgb]{0.627,0.126,0.941},
	showstringspaces=false,
	basicstyle=\tiny,
numberstyle=\small,
numbers=right,
	stepnumber=1,
	numbersep=10pt,
	tabsize=1,
	breaklines=true,
	prebreak = \raisebox{0ex}[0ex][0ex]{\ensuremath{\hookleftarrow}},
	breakatwhitespace=false,
	%aboveskip={1.5\baselineskip},
  columns=fixed,
  upquote=true,
  extendedchars=true,
 frame=single,
}

\lstdefinestyle{command}{
backgroundcolor=\color{yellow},frame=shadowbox,
language=c,
	keywordstyle=\bfseries\ttfamily\color[rgb]{0,0,1},
	identifierstyle=\ttfamily,
	commentstyle=\color[rgb]{0.133,0.545,0.133},
	stringstyle=\ttfamily\color[rgb]{0.627,0.126,0.941},
	showstringspaces=false,
	basicstyle=\small,
numberstyle=\small,
numbers=right,
	stepnumber=1,
	numbersep=10pt,
	tabsize=1,
	breaklines=true,
	prebreak = \raisebox{0ex}[0ex][0ex]{\ensuremath{\hookleftarrow}},
	breakatwhitespace=false,
	%aboveskip={1.5\baselineskip},
  columns=fixed,
  upquote=true,
  extendedchars=true,
}

\lstset{
	language=c,
	keywordstyle=\bfseries\ttfamily\color[rgb]{0,0,1},
	identifierstyle=\ttfamily,
	commentstyle=\color[rgb]{0.133,0.545,0.133},
	stringstyle=\ttfamily\color[rgb]{0.627,0.126,0.941},
	showstringspaces=false,
	basicstyle=\small,
numberstyle=\small,
numbers=right,
	stepnumber=1,
	numbersep=10pt,
	tabsize=1,
	breaklines=true,
	prebreak = \raisebox{0ex}[0ex][0ex]{\ensuremath{\hookleftarrow}},
	breakatwhitespace=false,
	%aboveskip={1.5\baselineskip},
  columns=fixed,
  upquote=true,
  extendedchars=true,
 frame=single
 %backgroundcolor=\color{lbcolor},
}

\begin{document}
\title{Introdução às ferramentas de observação do sistema de computação\\ Traçado dinâmico recorrendo a DTrace em Solaris 11}

% author names and affiliations
% use a multiple column layout for up to three different
% affiliations
\author{{Filipe Oliveira}
Departamento de Informática\\
Universidade do Minho\\
Email: a57816@alunos.uminho.pt}
}



% make the title area
\maketitle

% As a general rule, do not put math, special symbols or citations
% in the abstract
%\begin{abstract}

%Neste estudo, analisamos a performance de kernels 
%\end{abstract}

% no keywords




% For peer review papers, you can put extra information on the cover
% page as needed:
% \ifCLASSOPTIONpeerreview
% \begin{center} \bfseries EDICS Category: 3-BBND \end{center}
% \fi
%
% For peerreview papers, this IEEEtran command inserts a page break and
% creates the second title. It will be ignored for other modes.



%\section{Introduction}
% no \IEEEPARstart
%This demo file is intended to serve as a ``starter file''
%for IEEE Computer Society conference papers produced under \LaTeX\ using
%IEEEtran.cls version 1.8b and later.
% You must have at least 2 lines in the paragraph with the drop letter
% (should never be an issue)
%I wish you the best of success.

%\hfill Filipe Oliveira
 
%\hfill 1 Março, 2016
\renewcommand{\abstract}{\textbf{\centering Introdução -- Contextualização da Ferramenta DTrace\break}}

\abstract{

A necessidade de recurso a ferramentas de traçado dinâmico como o DTrace está implicitamente associada à necessidade de recolha de informação dos sistemas de computação no seu todo -- via agregação, ou de processos/kernels específicos. Com o aumento da complexidade dos sistemas existem "comportamentos" incorretos de kernels que apenas podem ser observados através da instrumentação dos kernels no próprio sistema e recolha de dados estatísticos da mesma. Ora, essa instrumentação pode ser via amostragem (\say{sampling}) ou via traçado dinâmico (\say{tracing}). \par
A grande vantagem do recurso à ferramenta DTrace está associada à segunda forma de instrumentação (apesar de a ferramenta também nos permitir a recolha de dados de amostragem). O Dtrace consegue acoplar informação extremamente  distinta, contando ainda com a mais valia de um overhead mínimo na grande maioria das medições dos valores de informação, via \say{instrumentation points} ou \say{probes}. \par 
No sistema em estudo contamos com 94274 probes (organizadas por provider:module:function:name). De seguida apresenta-se a lista de providers disponíveis no mesmo, conjuntamente com um excerto da sua contextualização extraído do livro SPEC\footnote{Systems performance : enterprise and the cloud / Brendan Gregg.} e do guia online do Dtrace\footnote{\url{http://dtrace.org/guide/preface.html}} :
\begin{itemize}
\item \textbf{cpc} : CPU performance counters
\item \textbf{dtrace} provides several probes related to DTrace itself
\item \textbf{fbt }: kernel-level dynamic tracing
\item \textbf{fsinfo}  allows tracing of file system events across different file system types, with file information for each event
\item \textbf{io} : block device interface tracing (disk I/O)
\item \textbf{ip }: IP protocol events: send and receive
\item \textbf{iscsi} : iSCSI protocol events: connections, send and receive
\item \textbf{lockstat} : lock contention statistics, or to understand virtually any aspect of locking behavior
\item \textbf{mib} : makes available probes that correspond to counters in the illumos management information bases (MIBs)
\item \textbf{proc} : process-level events: create, exec, exit
\item \textbf{profile} : provides probes associated with a time-based interrupt firing every fixed, specified time interval
\item \textbf{sched} : kernel scheduling events
\item \textbf{sdt} : creates probes at sites that a software programmer has formally designated
\item \textbf{shadowfs} ZFS Shadow Migration events
\item \textbf{syscall} : system call trap table
\item \textbf{sysevent} : system events
\item \textbf{sysinfo} : system statistics
\item \textbf{tcp} : TCP protocol events: connections, send and receive
\item \textbf{udp} : UDP protocol events: connections, send and receive
\item \textbf{vm} : virtual memory statistics
\item \textbf{vminfo} : virtual memory statistics
\end{itemize}


É importante compreender o anteriormente enumerado para associar corretamente a informação que queremos recolher dos kernels/sistemas às probes disponibilizadas. No seguimento do discutido durante as componentes práticas da UCE de Engenharia de Sistemas de Computação foi então proposta a realização de alguns exercícios de ambientação com a ferramenta de medição DTrace, que passarei de seguida a solucionar. 
}

\newpage
\section{}

\epigraph{Fazer o traçado das chamadas ao sistema open() que deverá imprimir a seguinte informação por linha:
\begin{itemize}
\item nome do ficheiro executável e respetivos: PID do processo, UID do utilizador e GID do grupo.
\item Caminho absoluto para o ficheiro que for aberto.
\item A cadeia de carateres com as “flags” da chamada ao sistema open(), O\_RDONLY,
O\_WRONLY, O\_RDWR, O\_APPEND, O\_CREAT
\item O Valor de retorno de chamada de sistema
\end{itemize}
\\
Testar o programa com as hipóteses que seguem:
\begin{itemize}
\item cat /etc/inittab $>$ /tmp/test
\item cat /etc/inittab $>>$ /tmp/test
\item cat /etc/inittab $|$ tee /tmp/test
\item cat /etc/inittab $|$ tee -a /tmp/test
\end{itemize}
\\
\textbf{Opcional}: Modificar o programa para que apenas os ficheiros com "/etc" no caminho sejam detetados.
}

Tal como referido no enunciado, em Solaris 11 a chamada de sistema \textbf{open()} foi substituído por outra mais genérico \textbf{openat()}. Analisando a assinatura da função openat()\footnote{\url{https://docs.oracle.com/cd/E26502_01/html/E28556/gkxro.html}}:

\begin{lstlisting}
int openat(int fildes, const char *path, int oflag, /* mode_t mode */);
\end{lstlisting}


 necessitamos então de associar a informação que necessitamos a probes disponíveis no sistema através do comando:
 \begin{lstlisting}[basicstyle=\scriptsize]
 a57816@solaris11:/share/jade/a57816/ESC_DTRACE$ dtrace -l  -f openat* 
   ID   PROVIDER            MODULE                          FUNCTION NAME
 1583    syscall                                              openat entry
 1584    syscall                                              openat return
 1585    syscall                                            openat64 entry
 1586    syscall                                            openat64 return
30132        fbt           genunix                          openat32 entry
30133        fbt           genunix                          openat32 return
30134        fbt           genunix                          openat64 entry
30135        fbt           genunix                          openat64 return
30649        fbt           genunix                            openat entry
30650        fbt           genunix                            openat return
a57816@solaris11:/share/jade/a57816/ESC_DTRACE$
 \end{lstlisting}

Desta forma confirmamos a existência das probes necessárias para a correcta realização do enunciado, sedo estas as de id 1583 a 1586 no nosso sistema de computação, escolha que iremos justificar de seguida.
Ora, os 4 primeiros campos requeridos ( nome do ficheiro executável e respetivos: PID do processo, UID do utilizador e GID do grupo) podem ser obtidos através das variáveis acessíveis na ferramenta DTrace \textbf{execname, pid, uid, gid}, disponíveis em qualquer das probes seleccionadas anteriormente. Contudo, o caminho absoluto para o ficheiro que for aberto pode apenas ser acedido através da variável \textbf{arg1} nas probes  \textbf{syscall::openat:entry} e \textbf{syscall:openat64:entry}, sendo ressalvada ainda a necessidade de copiar a string que contém o caminho absoluto da zona de memória do utilizador para o kernel. Das premissas anteriores sabemos que precisaremos da seguinte regra no nosso script exec1.d:
 \begin{lstlisting}
syscall::openat*:entry {
  self->pathname =copyinstr(arg1);
  .....
  .....
  .....
}
 \end{lstlisting}

Por forma a obtermos a cadeia de carateres com as “flags” da chamada ao sistema open(), O\_RDONLY,
O\_WRONLY, O\_RDWR, O\_APPEND, O\_CREAT, necessitamos de aceder aos valores do terceiro argumento das system calls \textbf{openat()} e \textbf{openat64()}.  Ora, tal informação será acedida também através das duas probes anteriormente descritas na variável arg2(terceira variável das funções), sendo necessário proceder à correcta interpretação e posterior impressão dos seus valores. A solução encontrada passa por guardar a cadeia de caracteres numa variável local \textbf{self-$>$flags = arg2} e proceder posteriormente à impressão. \par 
Relativamente à impressão das flags podemos dividir a mesma em duas grandes porções. Temos de uma lado o modo de abertura relativos às permissões (O\_RDONLY,O\_WRONLY, O\_RDWR) e as restantes duas flags do nosso interesse (O\_APPEND, O\_CREAT). Relativamente ao modo de abertura podemos considerar que temos apenas de realizar duas verificações, tendo em conta que os condicionais em DTrace se expressam pelo operador ternário. Assim, resolvemos a primeiro porção do problema com a seguinte expressão
\begin{lstlisting}
  /*if its not O_WRONLY or O_RDWR then its implicitly O_RDONLY */
 printf( "%-9s" , self->flags & O_WRONLY ? " O_WRONLY " : self->flags & O_RDWR ? " O_RDWR " : " O_RDONLY " ); 
  \end{lstlisting}
  
 Uma vez que para o ficheiro ser aberto em modo \textbf{O\_WRONLY} o \& lógico entre self-$>$flags e  \textbf{O\_WRONLY} terá que retornar o valor  \textbf{O\_WRONLY}. O mesmo se aplica para a flag de abertura  \textbf{O\_RDWR}. Caso a cadeia de caracteres não corresponder positivamente a nenhuma das anteriores verificações, implicará obrigatoriamente a abertura do ficheiro em modo  \textbf{O\_RDONLY} (modo de leitura).\par 
 Relativamente aos modos  \textbf{O\_APPEND} e  \textbf{O\_CREAT} as verificações são feitas separadamente pelas seguintes expressões:
 \begin{lstlisting}
  printf( "%-9s" , self->flags & O_APPEND ? "| O_APPEND " : "" );
  printf( "%-9s" , self->flags & O_CREAT ? "| O_CREAT " : "" );
   \end{lstlisting}
\par 
Contudo, o valor de retorno de chamada de sistema não pode ser acedido através das duas probes anteriores, dado que, no início da system call, ainda não existe a informação relativa ao sucesso ou insucesso da mesma e respectivo descritor de ficheiro em caso de sucesso. Assim, necessitamos de acrescentar uma outra regra ao nosso ficheiro exec1.d que, aquando do término das system calls \textbf{openat()} e \textbf{openat64()}, verifica o valor retornado na variável \textbf{fildes}, acessível através da variável DTrace arg1.\par 
Adicionando apenas uma regra para aquando do início da execução do nosso script imprimir o cabeçalho ficamos com o seguinte ficheiro final:
 \begin{lstlisting}[basicstyle=\scriptsize]
 #!/usr/sbin/dtrace -s

/*
 ********************************************************************************
 *   Copyright(C) 2016 Filipe Oliveira
 *   HPC Group, Computer Science Dpt.
 *   University of Minho
 *   All Rights Reserved.
 ********************************************************************************
 *   Content : simple openat and openat64 system calls tracer
 *     
 ********************************************************************************/

#pragma D option quiet

dtrace:::BEGIN {
  printf("%-10s%-8s%-8s%-8s%-50s%-27s%-5s\n",  "EXEC" , "PID", "UID" , "GID", "ABS PATH" , "FLAGS", "RETURNED VALUE");
}

/* will catch openat and openat64 */
syscall::openat*:entry {
  self->pathname =copyinstr(arg1);
  self->flags = arg2;
}

/* will catch openat and openat64 */
syscall::openat*:return
{
  printf("%-10s%-8d%-8d%-8d%-50s", execname, pid, uid, gid, self->pathname);
  /*if its not O_WRONLY or O_RDWR then its implicitly O_RDONLY */
  printf( "%-9s" , self->flags & O_WRONLY ? " O_WRONLY " : self->flags & O_RDWR ? " O_RDWR " : " O_RDONLY " ); 
  printf( "%-9s" , self->flags & O_APPEND ? "| O_APPEND " : "" );
  printf( "%-9s" , self->flags & O_CREAT ? "| O_CREAT " : "" );
  printf("%5i\n", arg1); 
}


 \end{lstlisting}

\subsection{Resultados de execução dos comandos exemplo}
Podemos agora executar os 4 comandos requeridos, apresentando de seguida os resultados de execução:
\subsubsection{}

 \begin{lstlisting}[style=command]
 cat /etc/inittab  >  /tmp/test1
   \end{lstlisting}
   
\par 
\begin{lstlisting}[style=esc]
EXEC      PID     UID     GID     ABS PATH                                          FLAGS                      RETURNED VALUE
bash      22658   29220   5000    /tmp/test1                                         O_WRONLY          | O_CREAT     4
cat       22658   29220   5000    /var/ld/ld.config                                  O_RDONLY                      -1
cat       22658   29220   5000    /lib/libc.so.1                                     O_RDONLY                       3
cat       22658   29220   5000    /usr/lib/locale/en_US.UTF-8/en_US.UTF-8.so.3       O_RDONLY                       3
cat       22658   29220   5000    /usr/lib/locale/en_US.UTF-8/methods_unicode.so.3   O_RDONLY                       3
cat       22658   29220   5000    /etc/inittab                                       O_RDONLY                       3
 \end{lstlisting}

\subsubsection{}

 \begin{lstlisting}[style=command]
 cat /etc/inittab  >>  /tmp/test1
   \end{lstlisting}
   Atente na primeira linha retornada, na flag O\_APPEND tal como previsível:
\par 
\begin{lstlisting}[style=esc]
EXEC      PID     UID     GID     ABS PATH                                          FLAGS                      RETURNED VALUE
bash      22662   29220   5000    /tmp/test1                                         O_WRONLY | O_APPEND | O_CREAT     4
cat       22662   29220   5000    /var/ld/ld.config                                  O_RDONLY                      -1
cat       22662   29220   5000    /lib/libc.so.1                                     O_RDONLY                       3
cat       22662   29220   5000    /usr/lib/locale/en_US.UTF-8/en_US.UTF-8.so.3       O_RDONLY                       3
cat       22662   29220   5000    /usr/lib/locale/en_US.UTF-8/methods_unicode.so.3   O_RDONLY                       3
cat       22662   29220   5000    /etc/inittab                                       O_RDONLY                       3
 \end{lstlisting}
 
\subsubsection{}

 \begin{lstlisting}[style=command]
cat /etc/inittab | tee /tmp/test1
   \end{lstlisting}
\par 
\begin{lstlisting}[style=esc]
EXEC      PID     UID     GID     ABS PATH                                          FLAGS                      RETURNED VALUE
tee       22666   29220   5000    /var/ld/ld.config                                  O_RDONLY                      -1
tee       22666   29220   5000    /lib/libc.so.1                                     O_RDONLY                       3
tee       22666   29220   5000    /usr/lib/locale/en_US.UTF-8/en_US.UTF-8.so.3       O_RDONLY                       3
tee       22666   29220   5000    /usr/lib/locale/en_US.UTF-8/methods_unicode.so.3   O_RDONLY                       3
tee       22666   29220   5000    /tmp/test1                                         O_WRONLY          | O_CREAT     3
cat       22665   29220   5000    /var/ld/ld.config                                  O_RDONLY                      -1
cat       22665   29220   5000    /lib/libc.so.1                                     O_RDONLY                       3
cat       22665   29220   5000    /usr/lib/locale/en_US.UTF-8/en_US.UTF-8.so.3       O_RDONLY                       3
cat       22665   29220   5000    /usr/lib/locale/en_US.UTF-8/methods_unicode.so.3   O_RDONLY                       3
cat       22665   29220   5000    /etc/inittab                                       O_RDONLY                       3
 \end{lstlisting}

\subsubsection{}

 \begin{lstlisting}[style=command]
cat /etc/inittab | tee -a /tmp/test1
   \end{lstlisting}
\par 
\begin{lstlisting}[style=esc]
EXEC      PID     UID     GID     ABS PATH                                          FLAGS                      RETURNED VALUE
tee       22656   29220   5000    /var/ld/ld.config                                  O_RDONLY                      -1
tee       22656   29220   5000    /lib/libc.so.1                                     O_RDONLY                       3
tee       22656   29220   5000    /usr/lib/locale/en_US.UTF-8/en_US.UTF-8.so.3       O_RDONLY                       3
tee       22656   29220   5000    /usr/lib/locale/en_US.UTF-8/methods_unicode.so.3   O_RDONLY                       3
tee       22656   29220   5000    /tmp/test1                                         O_WRONLY | O_APPEND | O_CREAT     3
cat       22655   29220   5000    /var/ld/ld.config                                  O_RDONLY                      -1
cat       22655   29220   5000    /lib/libc.so.1                                     O_RDONLY                       3
cat       22655   29220   5000    /usr/lib/locale/en_US.UTF-8/en_US.UTF-8.so.3       O_RDONLY                       3
cat       22655   29220   5000    /usr/lib/locale/en_US.UTF-8/methods_unicode.so.3   O_RDONLY                       3
cat       22655   29220   5000    /etc/inittab                                       O_RDONLY                       3
 \end{lstlisting}

\subsection{Resolução do exercício opcional}
Para modificar o programa para que apenas os ficheiros com "/etc" no caminho sejam detetados, necessitamos apenas de adicionar o predicado \textbf{$/strstr(self->pathname,"/etc") != NULL/$} às duas probes detectadas por \textbf{syscall::openat*:return}, produzindo o ficheiro \textbf{ex1\_opt.d}:

 \begin{lstlisting}[basicstyle=\scriptsize]
 #!/usr/sbin/dtrace -s

/*
 ********************************************************************************
 *   Copyright(C) 2016 Filipe Oliveira
 *   HPC Group, Computer Science Dpt.
 *   University of Minho
 *   All Rights Reserved.
 ********************************************************************************
 *   Content : simple openat and openat64 system calls tracer 
 *             with /etc/ on its pathname
 *     
 ********************************************************************************/

#pragma D option quiet

dtrace:::BEGIN {
  printf("%-10s%-8s%-8s%-8s%-30s%-27s%-5s\n",  "EXEC" , "PID", "UID" , "GID", "ABS PATH" , "FLAGS", "RETURNED VALUE");
}

/* will catch openat and openat64 */
syscall::openat*:entry
{
  self->pathname =copyinstr(arg1);
  self->flags = arg2;
}

/* will catch openat and openat64 */
syscall::openat*:return

/strstr(self->pathname,"/etc") != NULL/
{
  printf("%-10s%-8d%-8d%-8d%-30s", execname, pid, uid, gid, self->pathname);
  /*if its not O_WRONLY or O_RDWR then its implicitly O_RDONLY */
  printf( "%-9s" , self->flags & O_WRONLY ? " O_WRONLY " : self->flags & O_RDWR ? " O_RDWR " : " O_RDONLY " ); 
  printf( "%-9s" , self->flags & O_APPEND ? "| O_APPEND " : "" );
  printf( "%-9s" , self->flags & O_CREAT ? "| O_CREAT " : "" );
  printf("%5i\n", arg1);
}
 \end{lstlisting}

Com o seguinte exemplo de retorno de execução:

\begin{lstlisting}[style=esc]
EXEC      PID     UID     GID     ABS PATH                      FLAGS                      RETURNED VALUE
cat       22676   29220   5000    /etc/acct                      O_RDONLY                       3
cat       22676   29220   5000    /etc/aliases                   O_RDONLY                       3
cat       22676   29220   5000    /etc/amd64                     O_RDONLY                       3
cat       22676   29220   5000    /etc/anthy                     O_RDONLY                       3
cat       22676   29220   5000    /etc/apache2                   O_RDONLY                       3
cat       22676   29220   5000    /etc/auto_home                 O_RDONLY                       3
cat       22676   29220   5000    /etc/auto_master               O_RDONLY                       3
cat       22676   29220   5000    /etc/auto_share                O_RDONLY                       3
cat       22676   29220   5000    /etc/avahi                     O_RDONLY                       3
cat       22676   29220   5000    /etc/bash                      O_RDONLY                       3
cat       22676   29220   5000    /etc/bonobo-activation         O_RDONLY                       3
cat       22676   29220   5000    /etc/brand                     O_RDONLY                       3
cat       22676   29220   5000    /etc/brltty                    O_RDONLY                       3
cat       22676   29220   5000    /etc/certs                     O_RDONLY                       3
cat       22676   29220   5000    /etc/compizconfig              O_RDONLY                       3
cat       22676   29220   5000    /etc/ConsoleKit                O_RDONLY                       3
cat       22676   29220   5000    /etc/cron.d                    O_RDONLY                       3
cat       22676   29220   5000    /etc/crypto                    O_RDONLY                       3
cat       22676   29220   5000    /etc/cups                      O_RDONLY                       3
cat       22676   29220   5000    /etc/dacf.conf                 O_RDONLY                       3
cat       22676   29220   5000    /etc/dat                       O_RDONLY                       3
cat       22676   29220   5000    /etc/datemsk                   O_RDONLY                       3
cat       22676   29220   5000    /etc/dbus-1                    O_RDONLY                       3
cat       22676   29220   5000    /etc/default                   O_RDONLY                       3
cat       22676   29220   5000    /etc/defaultrouter             O_RDONLY                       3
cat       22676   29220   5000    /etc/dev                       O_RDONLY                       3
cat       22676   29220   5000    /etc/devices                   O_RDONLY                       3
cat       22676   29220   5000    /etc/devlink.tab               O_RDONLY                       3
cat       22676   29220   5000    /etc/dfs                       O_RDONLY                       3
cat       22676   29220   5000    /etc/dhcp                      O_RDONLY                       3
cat       22676   29220   5000    /etc/dladm                     O_RDONLY                       3
cat       22676   29220   5000    /etc/drirc                     O_RDONLY                       3
cat       22676   29220   5000    /etc/driver                    O_RDONLY                       3
cat       22676   29220   5000    /etc/driver_aliases            O_RDONLY                       3
cat       22676   29220   5000    /etc/driver_classes            O_RDONLY                       3
cat       22676   29220   5000    /etc/dumpadm.conf              O_RDONLY                       3
cat       22676   29220   5000    /etc/dumpdates                 O_RDONLY                       3
cat       22676   29220   5000    /etc/emulexDiscConfig          O_RDONLY                       3
cat       22676   29220   5000    /etc/emulexRMConfig            O_RDONLY                       3
cat       22676   29220   5000    /etc/emulexRMOptions           O_RDONLY                       3
cat       22676   29220   5000    /etc/flash                     O_RDONLY                       3
cat       22676   29220   5000    /etc/fm                        O_RDONLY                       3
cat       22676   29220   5000    /etc/fonts                     O_RDONLY                       3
cat       22676   29220   5000    /etc/foomatic                  O_RDONLY                       3
cat       22676   29220   5000    /etc/format.dat                O_RDONLY                       3
cat       22676   29220   5000    /etc/fs                        O_RDONLY                       3
cat       22676   29220   5000    /etc/ftpd                      O_RDONLY                       3
cat       22676   29220   5000    /etc/ftpusers                  O_RDONLY                       3
cat       22676   29220   5000    /etc/gconf                     O_RDONLY                       3
cat       22676   29220   5000    /etc/gdm                       O_RDONLY                       3
cat       22676   29220   5000    /etc/gnome-vfs-2.0             O_RDONLY                       3
cat       22676   29220   5000    /etc/gnome-vfs-mime-magic      O_RDONLY                       3
cat       22676   29220   5000    /etc/gnu                       O_RDONLY                       3
cat       22676   29220   5000    /etc/group                     O_RDONLY                       3
cat       22676   29220   5000    /etc/gss                       O_RDONLY                       3
cat       22676   29220   5000    /etc/gtk-2.0                   O_RDONLY                       3
cat       22676   29220   5000    /etc/hal                       O_RDONLY                       3
cat       22676   29220   5000    /etc/hba.conf                  O_RDONLY                       3
cat       22676   29220   5000    /etc/hostid                    O_RDONLY                       3
cat       22676   29220   5000    /etc/hosts                     O_RDONLY                       3
cat       22676   29220   5000    /etc/hp                        O_RDONLY                       3
cat       22676   29220   5000    /etc/ibadm                     O_RDONLY                       3
cat       22676   29220   5000    /etc/ima.conf                  O_RDONLY                       3
cat       22676   29220   5000    /etc/inet                      O_RDONLY                       3
cat       22676   29220   5000    /etc/inetd.conf                O_RDONLY                       3
 \end{lstlisting}

\newpage
\section{}

\epigraph{Mostrar para os processos que estão a correr no sistema as seguintes estatísticas, com valores obtidos durante cada iteração:
a)
\begin{itemize}
\item número de tentativas de abrir ficheiros existentes;
\item número de tentativas para criar ficheiros;
\item número de tentativas bem-sucedidas.
\end{itemize}
\\
b) Repetidamente, com um período (especificado em segundos) passado como argumentos da linha de comandos deve imprimir:
\begin{itemize}
\item  hora e dia atual em formato legível.
\item as estatísticas recolhidas por PID e respetivo o nome.
\end{itemize}
}
\par 
\subsection{}
Utilizando o script desenvolvido no exercício anterior, será necessário adicionar varáveis de agregação e contar cada tipo de abertura de ficheiro. Ora, por \textbf{\say{número de tentativas de abrir ficheiros existentes}} podemos considerar o número de aberturas de ficheiro sem a flag \textbf{O\_CREAT}, o que se traduz no predicado $/( arg2  \&  O\_CREAT) == 0 /$. Analogamente \textbf{\say{número de tentativas para criar ficheiros}} será traduzido no predicado $/ ( arg2  \&  O\_CREAT ) == O\_CREAT /$, ambos predicados a serem incluidos para as 2 probes descritas por \texbf{syscall::openat*:entry}.\par 
Resta-nos  traduzir \textbf{\say{número de tentativas bem-sucedidas}}. Ora, analisando a descrição da função openat() percebemos que apenas quando o descritor de ficheiro toma o valor -1 poderemos considerar que não foi bem sucedida a operação de abertura, que se traduz no predicado $//arg1 >= 0/$ para as 2 probes descritas por \textbf{syscall::openat*:return}.\par Definidas as iterações que estão ou não incluídas em cada regra resta-nos criar as variáveis que agregam a informação sendo estas:
\begin{lstlisting}
@successfull[ pid ];
@open_request[  pid ];
@create_request[  pid ];
 \end{lstlisting}
a serem alteradas a cada iteração (via count()) nas regras especificadas no ficheiro completo \textbf{ex2a.d}. Denote também que no início e fim do script são impressos o cabeçalho e as variáveis  que agregam a informação por \textbf{pid}:
\begin{lstlisting}
#!/usr/sbin/dtrace -s

/*
 ********************************************************************************
 *   Copyright(C) 2016 Filipe Oliveira
 *   HPC Group, Computer Science Dpt.
 *   University of Minho
 *   All Rights Reserved.
 ********************************************************************************
 *   Content : simple openat and openat64 system calls tracer and agregator by
 *             by pid and opening mode
 *     
 ********************************************************************************/

#pragma D option quiet
 
dtrace:::BEGIN {
 printf("*********************************************************************************\n");
 printf("openat and openat64 syscalls aggregator\n");
 printf("!=O_CREAT ::  opening files already in system\n");
 printf("  O_CREAT ::  opening files with flag to create\n"); 
 printf("    #SUCC ::  sucessfull openat and openat64 system calls\n");
 printf("* * * * * * * * * * * * * * * * * * * * * * * * * * * * * * * * * * * * * * * * *\n");
}

/* will catch openat and openat64 number of tries to create file */
syscall::openat*:entry 
/( arg2 & O_CREAT) == 0 /
{
  @open_request[ pid ] = count();
}

/* will catch openat and openat64 number of tries to create file */
syscall::openat*:entry 
/ ( arg2 & O_CREAT ) == O_CREAT /
{
  @create_request[ pid ] = count();
}

/* will catch openat and openat64 sucessfull file open
 * from linux man: " On success, openat() returns a new file descriptor. 
 * On error, -1 is returned and errno is set to indicate the error."  */
syscall::openat*:return
/arg1 >= 0/
{
  @successfull[ pid ] = count();
}

dtrace:::END {
 printf("* * * * * * * * * * * * * * * * * * * * * * * * * * * * * * * * * * * * * * * * *\n");
 printf( "%-6s\t%10s\t%10s\t%10s\n", "pid", "!=O_CREAT", "O_CREAT","#SUCC" );
 printf("*********************************************************************************\n");
 printa( "%6d\t%@10d\t%@10d\t%@10d\n", @open_request, @create_request, @successfull );
  clear(@open_request);
  clear(@create_request);
  clear(@successfull);
}
 \end{lstlisting}

Atente no exemplo do output do script:

\begin{lstlisting}[basicstyle=\scriptsize]
a57816@solaris11:/share/jade/a57816/ESC_DTRACE$ ./ex2a.d 
*********************************************************************************
openat and openat64 syscalls aggregator
!=O_CREAT ::  opening files already in system
  O_CREAT ::  opening files with flag to create
    #SUCC ::  sucessfull openat and openat64 system calls
* * * * * * * * * * * * * * * * * * * * * * * * * * * * * * * * * * * * * * * * *
^C
* * * * * * * * * * * * * * * * * * * * * * * * * * * * * * * * * * * * * * * * *
pid      !=O_CREAT         O_CREAT           #SUCC
*********************************************************************************
  1351           1               0               1
  1447           3               0               3
 22701           4               0               3
 22487           5               0               5
 22699           7               0               4
 22700          11               0               8
 22698         196               0             190
 \end{lstlisting}

\par 
\subsection{}
Relativamente às estatísticas agregadas e impressas repetidamente, com um período (especificado em segundos) passado como argumentos da linha de comandos, podemos recorrer à variável \textbf{walltimestamp} por forma a obter o valor da hora e dia atual em formato legível. Para obtermos o valor em segundos da linha de comandos resta-nos verificar o valor da variável \textbf{\$1} e recorrer a \textbf{tick-\$1s} para imprimir a um ritmo de \$1 segundos.\par 
Por forma a imprimirmos apenas os valores agregados entre duas medições necessitamos de limpar os valores presentes nas variáveis agregadas. Tal é realizado recorrendo ao método \textbf{trunc}:
\begin{lstlisting}
  trunc(@open_request);
  trunc(@create_request);
  trunc(@successfull);
 \end{lstlisting}
Dado considerar que esta alínea do exercício ser um complemento do primeiro requisito foram adicionadas as variáveis:

\begin{lstlisting}
@all_successfull[ pid, execname ];
@all_open_request[  pid, execname ];
@all_create_request[  pid, execname ];
 \end{lstlisting}
 que mantêm a possibilidade do utilizador visualizar no término da script os valores completos agregados.
Assim, o ficheiro \textbf{ex2b.d} apresenta o seguinte formato:

\begin{lstlisting}
#!/usr/sbin/dtrace -s

/*
 ********************************************************************************
 *   Copyright(C) 2016 Filipe Oliveira
 *   HPC Group, Computer Science Dpt.
 *   University of Minho
 *   All Rights Reserved.
 ********************************************************************************
 *   Content : simple openat and openat64 system calls tracer and agregator by
 *             by pid and opening mode at a constant time rate passed by argumment
 *     
 ********************************************************************************/

#pragma D option quiet

dtrace:::BEGIN {
  printf("*********************************************************************************\n");
  printf("openat and openat64 syscalls aggregator by constant time rate passed by argumment\n");
  printf("\n TIME RATE %d seconds \n", $1);
  printf("START TIME %Y \n\n", walltimestamp);
  printf("!=O_CREAT ::  opening files already in system\n");
  printf("  O_CREAT ::  opening files with flag to create\n");
  printf("    #SUCC ::  sucessfull openat and openat64 system calls\n");
  printf("* * * * * * * * * * * * * * * * * * * * * * * * * * * * * * * * * * * * * * * * *\n");
  printf( "%-6s\t%-20s\t%10s\t%10s\t%10s\n", "pid", "execname", "!=O_CREAT", "O_CREAT","#SUCC" );
  printf("* * * * * * * * * * * * * * * * * * * * * * * * * * * * * * * * * * * * * * * * *\n");

}

/* will catch openat and openat64 number of tries to create file */
syscall::openat*:entry 
/( arg2 & O_CREAT) == 0 /
{
  @open_request[  pid, execname ] = count();
  @all_open_request[  pid, execname ] = count();
}

/* will catch openat and openat64 number of tries to create file */
syscall::openat*:entry 
/ ( arg2 & O_CREAT ) == O_CREAT /
{
  @create_request[  pid, execname ] = count();
  @all_create_request[  pid, execname ] = count();
}

/* will catch openat and openat64 sucessfull file open
 * from linux man: " On success, openat() returns a new file descriptor. 
 * On error, -1 is returned and errno is set to indicate the error."  */
syscall::openat*:return
/arg1 >= 0/
{
  @successfull[ pid, execname ] = count();
  @all_successfull[ pid, execname ] = count();
}

tick-$1s {
  printf("\n[ %20Y * * * * * * * * * * \n", walltimestamp);
  printa( "%6d\t%-20s\t%@10d\t%@10d\t%@10d\n", @open_request, @create_request, @successfull );
  trunc(@open_request);
  trunc(@create_request);
  trunc(@successfull);
  printf("                                         * * * * * * * * * * * * * * * * * * * *]\n");
}

dtrace:::END {
  printf("\n**************************** AGGREGATED RESULTS *********************************\n");
  printf("                               %20Y \n\n", walltimestamp);
  printa( "%6d\t%-20s\t%@10d\t%@10d\t%@10d\n", @all_open_request, @all_create_request, @all_successfull );
  printf("\n*********************************************************************************\n");
  clear(@all_open_request);
  clear(@all_create_request);
  clear(@all_successfull);
  clear(@open_request);
  clear(@create_request);
  clear(@successfull);
}

 \end{lstlisting}

Atente no exemplo do output do script:

\begin{lstlisting}[basicstyle=\scriptsize]
*********************************************************************************
openat and openat64 syscalls aggregator by constant time rate passed by argumment

 TIME RATE 1 seconds 
START TIME 2016 Apr 10 03:52:13 

!=O_CREAT ::  opening files already in system
  O_CREAT ::  opening files with flag to create
    #SUCC ::  sucessfull openat and openat64 system calls
* * * * * * * * * * * * * * * * * * * * * * * * * * * * * * * * * * * * * * * * *
pid     execname                 !=O_CREAT         O_CREAT           #SUCC
* * * * * * * * * * * * * * * * * * * * * * * * * * * * * * * * * * * * * * * * *

[ 2016 Apr 10 03:52:13 * * * * * * * * * * 
 22713  ex2b.d                           2               0               2
                                         * * * * * * * * * * * * * * * * * * * *]

[ 2016 Apr 10 03:52:14 * * * * * * * * * * 
   259  utmpd                            3               1               4
                                         * * * * * * * * * * * * * * * * * * * *]

[ 2016 Apr 10 03:52:15 * * * * * * * * * * 
                                         * * * * * * * * * * * * * * * * * * * *]

[ 2016 Apr 10 03:52:16 * * * * * * * * * * 
                                         * * * * * * * * * * * * * * * * * * * *]

[ 2016 Apr 10 03:52:17 * * * * * * * * * * 
                                         * * * * * * * * * * * * * * * * * * * *]

[ 2016 Apr 10 03:52:18 * * * * * * * * * * 
                                         * * * * * * * * * * * * * * * * * * * *]

[ 2016 Apr 10 03:52:19 * * * * * * * * * * 
                                         * * * * * * * * * * * * * * * * * * * *]

[ 2016 Apr 10 03:52:20 * * * * * * * * * * 
                                         * * * * * * * * * * * * * * * * * * * *]

[ 2016 Apr 10 03:52:21 * * * * * * * * * * 
                                         * * * * * * * * * * * * * * * * * * * *]

[ 2016 Apr 10 03:52:22 * * * * * * * * * * 
 22487  bash                             1               0               1
 22714  cat                             35               0              30
                                         * * * * * * * * * * * * * * * * * * * *]

[ 2016 Apr 10 03:52:23 * * * * * * * * * * 
                                         * * * * * * * * * * * * * * * * * * * *]
^C

**************************** AGGREGATED RESULTS *********************************
                               2016 Apr 10 03:52:24 

 22487  bash                             1               0               1
 22713  ex2b.d                           2               0               2
   259  utmpd                            3               1               4
 22714  cat                             35               0              30

*********************************************************************************
 \end{lstlisting}
 
 
 
\newpage
\section{Conclusão}
Tal como mencionado no início do presente caso de estudo a ferramenta DTrace mostra-se bastante útil e única em termos de funcionalidades quando necessitamos de agregar informação de vários processos/threads,etc. Ou seja, no contexto da computação paralela será extremamente interessante recorrer a esta ferramenta de traçado dinâmico na execução de algoritmos paralelos. \par 
Este foi apenas um trabalho introdutório mas permitiu demostrar a capacidade de recolher e ao mesmo tempo tratar dados de todo um sistema extremamente complexo e vasto com apenas uma ferramenta. O caso de estudo ultrapassa portanto os resultados obtidos pelas scripts geradas, prendendo-se uma vez mais com o desenvolvimento de capacidade  prática no uso da ferramenta, e envolvimento com métodos de tratamento de grandes volumes de dados, e análise de métricas de sistemas de computação de alta perfomance.\par Retrata sobretudo a capacidade analisar funcionalidades disponibilizadas e a sua correta aplicação na resolução de problemas de computação tendo sempre em conta o mínimo de alteração possível na performance dos kernels/sistemas a analisar.

\newpage
\appendix 
\section{Uma análise à política de escalonamento de zonas paralelas em OpenMP recorrendo à ferramenta Dtrace}
Retomando o trabalho prático 2, e respectiva análise de paralelismo em ambiente de memória partilhada,  via zonas paralelas OpenMP será interessante, como extra, analisar a influência dos diversos tipos de escalonamento OpenMP: 
\begin{itemize}
\item {\textbf{static} -- a maioria dos compiladores dividem o trabalho dos loops em $\frac{N\ #iteracoes}{ p\ # threads}$ por default, sendo o número de iterações distribuído uniformemente por thread OpenMP.\par
A título ilustrativo, suponha que existem 1000 iterações a serem distribuídas de forma estática por 4 threads OpenMP. O loop será repartido, em caso standard, da seguinte forma:
    \begin{figure}[H]
    \centering
    \includegraphics[width=0.5\columnwidth]{PNG/schedule_static.png}
        \caption{ Ilustração do escalonamento estático:}
    \label{fig:schedule_static}
    \end{figure}
    }

    Ora, esta poderá não ser a melhor forma de escalonamento, muito devido à potencial irregularidade de tempos de conclusão de iteração por cada thread. Em caso de trabalhos regulares esta opção representa o menor overhead do paralelismo. Para casos de um trabalho por iteração por thread com tempos irregulares de conclusão, teremos casos de \textbf{load imbalance}, casos para os quais as opções de escalonamento \textbf{dynamic} e textbf{guided} deverão representar uma melhor solução.\par 
        
\item \textbf{dynamic} -- O standard OpenMP providência duas formas de escalonamento dinâmico - \textbf{dynamic} e \textbf{guided}, sendo que o último será falado no ponto seguinte. Com escalonamento dinâmico, novas porções de iterações dos ciclos serão atribuídas às threads conforme o trabalho for sendo concluído, sendo essas porções de iterações fixas.\par

\item \textbf{guided} -- com escalonamento guided, novas porções de iterações dos ciclos serão atribuídas às threads conforme o trabalho for sendo concluído, sendo essas porções de iterações dependentes relativas ao número de iterações restantes.\par
\end{itemize}
 
Atente no seguinte excerto de código:
  \lstinputlisting[language=C]{../extra/ex2_v2.cpp} %input de um ficheiro

Que iremos executar para as 3 versões de escalonamento dinâmico especificadas anteriormente, por forma a recolhermos dados estatísticos que nos permitam concluir qual das 3 versões a melhor para o nosso kernel específico.\par
Relativamente às estatísticas teremos interesse em ter conhecimento do momento de criação e término das threads OpenMP, tempos on e off CPU, respectivo número de CPU onde a thread está alocada, assim como as interrupções voluntárias e forçadas e seus tipos.\par 
Assim, o ficheiro \textbf{threaded.d}, disponibilizado no contexto da disciplina, apresenta o seguinte formato:

  \lstinputlisting[]{../extra/threaded.d} %input de um ficheiro
  
  Com base nos valores impressos pela execução do script dtrace poderemos tratar posterior os dados por forma a obtermos uma representação visual, à semelhança do realizado no trabalho prático 2.
  
\end{document}


